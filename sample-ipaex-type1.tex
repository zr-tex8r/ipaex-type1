\documentclass[a4paper]{article}
\usepackage{CJKutf8,CJKspace,CJKpunct,ruby}
\setlength{\parindent}{0pt}
%% \begin{jaQuote}{<family>}...\end{jaQuote}
\newenvironment{jaQuote}[1]{%
  \begin{quote}\linespread{1.4}\selectfont
  \begin{CJK*}{UTF8}{#1}\CJKtilde
}{%
  \end{CJK*}\end{quote}%
}
%% \jaRuby{<base>}{<ruby>}
\newcommand{\jaRuby}[2]{{%
  \renewcommand{\rubysize}{0.5}%
  \renewcommand{\rubysep}{-0.12em}\rubyCJK
  \setlength{\baselineskip}{1.257em}\linespread{}%
  \ruby{#1}{#2}%
}}
%% sample text
\newcommand{\jaEllip}{\Unicode{"20}{"26}}
\newcommand{\jaSampleText}{%
  この~FAQ~リストは、よくある質問とその答を集め、
  役に立つようにしたものです。
  この~FAQ~リストの構造は、以前のものと比べて
  大幅に変更されています。%
  \jaRuby{新}{あたら}しい構造に関しては、
  「この~FAQ~の読み方とその構造」の
  項目を\jaRuby{参照}{さんしょう}して下さい。
}
\begin{document}

Family `\texttt{ipxm}' (IPAexMincho):
\begin{jaQuote}{ipxm}
  \jaSampleText
\end{jaQuote}

(also using \texttt{ipxm} for alphabet)
\begin{jaQuote}{ipxm}
  \fontfamily{ipxm}\selectfont
  pdf{\LaTeX}~と~CJK~パッケージで日本語!
  \par\punctstyle{plain}
  % Japanese "Ascii Art" --- one of the few places
  % where "hankaku-kana" is reasonably used
  ( ˘ω˘ )スヤァ\jaEllip
\end{jaQuote}

\sffamily
Family `\texttt{ipxg}' (IPAexGothic):
\begin{jaQuote}{ipxg}
  \jaSampleText
\end{jaQuote}

(also using \texttt{ipxg} for alphabet)
\begin{jaQuote}{ipxg}
  \fontfamily{ipxg}\selectfont
  pdf{\LaTeX}~と~CJK~パッケージで日本語!
  \par\punctstyle{plain}
  ( ˘ω˘ )スヤァ\jaEllip
\end{jaQuote}

\end{document}
